% Preamble
\documentclass[11pt]{article}

% Packages
\usepackage{amsmath}
\usepackage{titling}
\usepackage{array}
\usepackage{longtable}
\usepackage{graphicx}
\graphicspath{ {/} }

% Titles
\title{Memoria: Procesadores de Lenguaje - Lenguaje Tiny}
\author{Burgos Sosa, Rodrigo \and Cassin, Gina Andrea \and Estebán Velasco, Luis \and Rabbia, Santiago Elias}
\date{Curso 2024}

% Document
\begin{document}
\begin{titlepage}
    \centering
    {\Huge Memoria: Procesadores de Lenguaje - Lenguaje Tiny \par}
    \vspace{1cm}
    {\Large Fase 3: Desarrollo de constructores de ASTs \par}
    \vspace{2cm}
    {\Large Grupo G03: \par}
    {\Large Burgos Sosa Rodrigo, Cassin Gina Andrea, \par}
    {\Large Estebán Velasco Luis, Rabbia Santiago Elias \par}
    \vspace{2cm}
    {\Large Curso 2024 \par}
\end{titlepage}
\thispagestyle{empty}

    \newpage

    \section{Introducción}
        En el siguiente documento se expondrá una memoria sobre el desarrollo de constructores de ASTs descendentes y ascendentes sobre el lenguaje de programación Tiny.
        Se presentará una especificación abstracta del lenguaje mediante la enumeración de las signaturas de las funciones constructoras de ASTs, se especificará el constructor de ASTs mediante una gramática s-atribuida y se acondicionará para permitir la implementación descendente. También se especificará un procesamiento para imprimir los tokens del programa leído. 

\end{document}