% Preamble
\documentclass[11pt]{article}

% Packages
\usepackage{amsmath}
\usepackage{titling}
\usepackage{array}
\usepackage{longtable}
\usepackage{graphicx}
\graphicspath{ {/} }

% Titles
\title{Memoria: Procesadores de Lenguaje - Lenguaje Tiny}
\author{Burgos Sosa, Rodrigo \and Cassin, Gina Andrea \and Estebán Velasco, Luis \and Rabbia, Santiago Elias}
\date{Curso 2024}

% Document
\begin{document}
\begin{titlepage}
    \centering
    {\Huge Memoria: Procesadores de Lenguaje - Lenguaje Tiny \par}
    \vspace{1cm}
    {\Large Fase 2: Análisis sintáctico \par}
    \vspace{2cm}
    {\Large Grupo G03: \par}
    {\Large Burgos Sosa Rodrigo, Cassin Gina Andrea, \par}
    {\Large Estebán Velasco Luis, Rabbia Santiago Elias \par}
    \vspace{2cm}
    {\Large Curso 2024 \par}
\end{titlepage}
\thispagestyle{empty}

    \newpage

    \section{Introducción}
        En el siguiente documento se expondrá una memoria sobre el desarrollo de analizadores sintácticos aplicado sobre dos lenguajes de programación, Tiny y Tiny(0) - un subconjunto de Tiny.
        Se presentará una especificación sintáctica de ambos lenguajes, con acondicionamiento de la gramática para permitir la implementación de un analizador
        sintáctico descendente predictivo recursivo, y se proporcionarán los directores de cada regla de la gramática acondicionada para solamente Tiny(0).

    \section{Análisis sintáctico}
        Es un componente central de un procesador de lenguaje, ya que dicha estructura gramatical será la base para articular los subsecuentes
        procesamientos del lenguaje, este es, el procesamiento dirigido por la sintaxis. Esto es lo que se especificará a continuación:
    
        \subsection{Tiny(0)}
            \subsubsection{Especificación sintáctica (gramática)}
            \begin{itemize}
                \item programa $\rightarrow$ bloque
                \item bloque $\rightarrow$ \{ declaraciones instrucciones \}
            \end{itemize}
            \
            \begin{itemize}
                \item declaraciones $\rightarrow$ declaracionesAux \&\&
                \item declaraciones $\rightarrow$ $\varepsilon$
                \item declaracionesAux $\rightarrow$ declaracionesAux ; declaracionVar
                \item declaracionesAux $\rightarrow$ declaracionVar
                \item declaracionVar $\rightarrow$ tipo identificador
            \end{itemize}
            \ 
            \begin{itemize}
                \item tipo $\rightarrow$ int
                \item tipo $\rightarrow$ real
                \item tipo $\rightarrow$ bool
            \end{itemize}
            \ 
            \begin{itemize}
                \item instrucciones $\rightarrow$ instruccionesAux
                \item instrucciones $\rightarrow$ $\varepsilon$
                \item instruccionesAux $\rightarrow$ instruccionesAux ; instruccion
                \item instruccionesAux $\rightarrow$ instruccion
                \item instruccion $\rightarrow$ \verb|@| expr 
            \end{itemize}
            
            Las siguientes expresiones se construyen de acuerdo a las siguientes prioridades (de menor a mayor prioridad):
            \begin{enumerate}
                \item Operador de asignación, es binario e infijo. 
                \item Operadores relacionales, son binarios e infijos.
                \item +, - (binario)
                \item and, or
                \item $^{\ast}$, / son binarios e infijos.
                \item - (unario), not son binarios y prefijos.
            \end{enumerate}
            \
            \begin{itemize}
                \item expr $\rightarrow$ e0
                \item e0 $\rightarrow$ e1 = e0 \textit{(asocia por derecha)}
                \item e0 $\rightarrow$ e1
                \item e1 $\rightarrow$ e1 op1 e2 \textit{(asocia por izquierda)}
                \item e1 $\rightarrow$ e2
                \item e2 $\rightarrow$ e2 + e3 \textit{(asocia por izquierda)}
                \item e2 $\rightarrow$ e3 - e3 \textit{(no asocia)}
                \item e2 $\rightarrow$ e3
                \item e3 $\rightarrow$ e4 and e3 \textit{(asocia por derecha)}
                \item e3 $\rightarrow$ e4 or e4 \textit{(no asocia)}
                \item e3 $\rightarrow$ e4
                \item e4 $\rightarrow$ e4 op4 e5 \textit{(asocia por izquierda)}
                \item e4 $\rightarrow$ e5
                \item e5 $\rightarrow$ op5 e5 \textit{(prefijo)}
                \item e5 $\rightarrow$ e6
                \item e6 $\rightarrow$ ( e0 )
                \item e6 $\rightarrow$ literalEntero
                \item e6 $\rightarrow$ literalReal
                \item e6 $\rightarrow$ true
                \item e6 $\rightarrow$ false
                \item e6 $\rightarrow$ identificador
                \item op1 $\rightarrow$ $<$
                \item op1 $\rightarrow$ $>$
                \item op1 $\rightarrow$ $<=$
                \item op1 $\rightarrow$ $>=$
                \item op1 $\rightarrow$ $==$
                \item op1 $\rightarrow$ $!=$
                \item op4 $\rightarrow$ $^{\ast}$
                \item op4 $\rightarrow$ $/$
                \item op5 $\rightarrow$ $-$
                \item op5 $\rightarrow$ not
            \end{itemize}
            \subsubsection{Acondicionamiento de la gramática (descendente predictivo recursivo)}
            Por notación, la eliminación de recursión por izquierda será denotada comenzando el nombre por $rec$ y la factorización por $fac$.  
            \begin{itemize}
                \item programa $\rightarrow$ bloque
                \item bloque $\rightarrow$ \{ declaraciones instrucciones \}
            \end{itemize}
            \
            \begin{itemize}
                \item declaraciones $\rightarrow$ declaracionesAux \&\&
                \item declaraciones $\rightarrow$ $\varepsilon$
                \item declaracionesAux $\rightarrow$ declaracionVar recDeclaracionesAux
                \item recDeclaracionesAux $\rightarrow$ ; declaracionVar recDeclaracionesAux
                \item recDeclaracionesAux $\rightarrow$ $\varepsilon$
                \item declaracionVar $\rightarrow$ tipo identificador
            \end{itemize}
            \ 
            \begin{itemize}
                \item tipo $\rightarrow$ int
                \item tipo $\rightarrow$ real
                \item tipo $\rightarrow$ bool
            \end{itemize}
            \ 
            \begin{itemize}
                \item instrucciones $\rightarrow$ instruccionesAux
                \item instrucciones $\rightarrow$ $\varepsilon$
                \item instruccionesAux $\rightarrow$ instruccion recInstruccionesAux 
                \item recInstruccionesAux  $\rightarrow$ ; instruccion recInstruccionesAux 
                \item recInstruccionesAux  $\rightarrow$ $\varepsilon$ 
                \item instruccion $\rightarrow$ \verb|@| expr 
            \end{itemize}
            \ 
            \begin{itemize}
                \item expr $\rightarrow$ e0
                \item e0 $\rightarrow$ e1 facE0
                \item facE0 $\rightarrow$ = e0
                \item facE0 $\rightarrow$ $\varepsilon$
                \item e1 $\rightarrow$ e2 recE1
                \item recE1 $\rightarrow$ op1 e2 recE1
                \item recE1 $\rightarrow$ $\varepsilon$
                \item e2 $\rightarrow$ e3 facE2 recE2
                \item recE2 $\rightarrow$ + e3 recE2
                \item recE2 $\rightarrow$ $\varepsilon$
                \item facE2  $\rightarrow$ - e3
                \item facE2  $\rightarrow$ $\varepsilon$
                \item e3 $\rightarrow$ e4 facE3
                \item facE3 $\rightarrow$ and e3
                \item facE3 $\rightarrow$ or e4
                \item facE3 $\rightarrow$ $\varepsilon$
                \item e4 $\rightarrow$ e5 recE4
                \item recE4 $\rightarrow$ op4 e5 recE4
                \item recE4 $\rightarrow$ $\varepsilon$
                \item e5 $\rightarrow$ op5 e5
                \item e5 $\rightarrow$ e6
                \item e6 $\rightarrow$ ( e0 )
                \item e6 $\rightarrow$ literalEntero
                \item e6 $\rightarrow$ literalReal
                \item e6 $\rightarrow$ true
                \item e6 $\rightarrow$ false
                \item e6 $\rightarrow$ identificador
                \item op1 $\rightarrow$ $<$
                \item op1 $\rightarrow$ $>$
                \item op1 $\rightarrow$ $<=$
                \item op1 $\rightarrow$ $>=$
                \item op1 $\rightarrow$ $==$
                \item op1 $\rightarrow$ $!=$
                \item op4 $\rightarrow$ $^{\ast}$
                \item op4 $\rightarrow$ $/$
                \item op5 $\rightarrow$ $-$
                \item op5 $\rightarrow$ not
            \end{itemize}
            \subsubsection{Directores de cada regla de la gramática condicionada}
            \begin{longtable}{|p{6cm}|p{4cm}|p{3cm}|}
                \hline
                \textbf{REGLA} & \textbf{DIRECTORES} & \textbf{ANULABLE} \\
                \hline
                programa $\rightarrow$ bloque & \{ & no \\
                \hline
                bloque $\rightarrow$ \{ declaraciones instrucciones \} & \{ & no \\
                \hline
                declaraciones $\rightarrow$ declaracionesAux \&\& & tipo & no \\
                \hline
                declaraciones $\rightarrow$ $\varepsilon$ & & si \\
                \hline
                declaracionesAux $\rightarrow$ declaracionVar recDeclaracionesAux & tipo & no \\
                \hline
                recDeclaracionesAux $\rightarrow$ ; declaracionVar & tipo & no \\
                \hline
                recDeclaracionesAux $\rightarrow$ $\varepsilon$ &  & si \\
                \hline
                declaracionVar  $\rightarrow$ tipo identificador & tipo & no \\
                \hline
                tipo $\rightarrow$ int & int & no \\
                \hline
                tipo $\rightarrow$ real & real & no \\
                \hline
                tipo $\rightarrow$ bool & bool & no \\
                \hline
                instrucciones $\rightarrow$ instruccionesAux & \verb|@| & no \\
                \hline
                instrucciones $\rightarrow$ $\varepsilon$ & & si \\
                \hline
                instruccionesAux $\rightarrow$ instruccion recInstruccionesAux & \verb|@| & no \\
                \hline
                recInstruccionesAux $\rightarrow$ ; instruccion recInstruccionesAux & ; & no \\
                \hline
                recInstruccionesAux $\rightarrow$ $\varepsilon$ &  & si \\
                \hline
                instruccion $\rightarrow$ \verb|@| expr & \verb|@| & no \\
                \hline
                expr $\rightarrow$ e0 & ( - not literalEntero literalReal true false identificador & no \\
                \hline
                e0 $\rightarrow$ e1 facE0 & ( - not literalEntero literalReal true false identificador & no \\
                \hline
                facE0 $\rightarrow$ = e0 & = & no \\
                \hline
                facE0 $\rightarrow$ $\varepsilon$ &  & si \\
                \hline
                e1 $\rightarrow$ e2 recE1 & ( - not literalEntero literalReal true false identificador & no \\
                \hline
                recE1 $\rightarrow$ op1 e2 recE1 & $< \ \ > \ \ <= \ \ >= \ \ == \ \  !=$ & no \\
                \hline
                recE1 $\rightarrow$ $\varepsilon$ &  & si \\
                \hline
                e2 $\rightarrow$ e3 facE2 recE2 & ( - not literalEntero literalReal true false identificador & no \\
                \hline
                recE2 $\rightarrow$ + e3 recE2 & + & no \\
                \hline
                recE2 $\rightarrow$ $\varepsilon$ &  & si \\
                \hline
                facE2  $\rightarrow$ e3 & - & no \\
                \hline
                facE2  $\rightarrow$ $\varepsilon$ &  & si \\
                \hline
                e3 $\rightarrow$ e4 facE3 & ( - not literalEntero literalReal true false identificador & no \\
                \hline
                facE3 $\rightarrow$ and e3 & and & no \\
                \hline
                facE3 $\rightarrow$ or e4 & or & no \\
                \hline
                facE3 $\rightarrow$ $\varepsilon$ &  & si \\
                \hline
                e4 $\rightarrow$ e5 recE4 & ( - not literalEntero literalReal true false identificador & no \\
                \hline
                recE4 $\rightarrow$ op4 e5 recE4 & $^{\ast}$ / & no \\
                \hline
                recE4 $\rightarrow$ $\varepsilon$ &  & si \\
                \hline
                e5 $\rightarrow$ op5 e5 & - not & no \\
                \hline
                e5 $\rightarrow$ e6 & ( - not literalEntero literalReal true false identificador & no \\
                \hline
                e6 $\rightarrow$ ( e0 ) & ( & no \\
                \hline
                e6 $\rightarrow$ literalEntero & literalEntero & no \\
                \hline
                e6 $\rightarrow$ literalReal & literalEntero & no \\
                \hline
                e6 $\rightarrow$ true & true & no \\
                \hline
                e6 $\rightarrow$ false & false & no \\
                \hline
                e6 $\rightarrow$ identificador & identificador & no \\
                \hline
                op1 $\rightarrow$ $<$ & $<$ & no \\
                \hline
                op1 $\rightarrow$ $>$ & $>$ & no \\
                \hline
                op1 $\rightarrow$ $<=$ & $<=$ & no \\
                \hline
                op1 $\rightarrow$ $>=$ & $>=$ & no \\
                \hline
                op1 $\rightarrow$ $==$ & $==$ & no \\
                \hline
                op1 $\rightarrow$ $!=$ & $!=$ & no \\
                \hline
                op4 $\rightarrow$ $^{\ast}$ & $^{\ast}$ & no \\
                \hline
                op4 $\rightarrow$ $/$ & $/$ & no \\
                \hline
                op5 $\rightarrow$ - & - & no \\
                \hline
                op5 $\rightarrow$ not & not & no \\
                \hline
            \end{longtable}
        \subsection{Tiny}
        \subsubsection{Especificación sintáctica (gramática)}
        \begin{itemize}
            \item programa $\rightarrow$ bloque
            \item bloque $\rightarrow$ \{ declaraciones instrucciones \}
        \end{itemize}
        \
        \begin{itemize}
            \item declaraciones $\rightarrow$ declaracionesAux \&\&
            \item declaraciones $\rightarrow \varepsilon$
            \item declaracionesAux $\rightarrow$ declaracionesAux ; declaracion
            \item declaracionesAux $\rightarrow$ declaracion
            \item declaracion $\rightarrow$ declaracionVar
            \item declaracion $\rightarrow$ declaracionTipo
            \item declaracion $\rightarrow$ declaracionProc
            \item declaracionVar $\rightarrow$ tipo0 identificador
            \item declaracionTipo $\rightarrow$ type tipo0 identificador
            \item declaracionProc $\rightarrow$ proc identificador paramsFormales bloque
            \item paramsFormales $\rightarrow$ ( paramsFormalesAux )
            \item paramsFormalesAux $\rightarrow$ paramsFormalesLista
            \item paramsFormalesAux $\rightarrow$ $\varepsilon$
            \item paramsFormalesLista $\rightarrow$ paramsFormalesLista , param 
            \item paramsFormalesLista $\rightarrow$ param
            \item param $\rightarrow$ tipo0 referencia identificador
            \item referencia $\rightarrow$ \&
            \item referencia $\rightarrow \varepsilon$
        \end{itemize}
        \
        \begin{itemize}
            \item tipo0 $\rightarrow$ tipo0 [ literalEntero ]
            \item tipo0 $\rightarrow$ tipo1
            \item tipo1 $\rightarrow$ \^{} tipo1
            \item tipo1 $\rightarrow$ tipo2
            \item tipo2 $\rightarrow$ struct \{ listaCampos \}
            \item listaCampos $\rightarrow$ listaCampos , campo
            \item listaCampos $\rightarrow$ campo
            \item campo $\rightarrow$ tipo0 identificador
            \item tipo2 $\rightarrow$ int
            \item tipo2 $\rightarrow$ real
            \item tipo2 $\rightarrow$ bool
            \item tipo2 $\rightarrow$ string
            \item tipo2 $\rightarrow$ identificador
        \end{itemize}
        \
        \begin{itemize}
            \item instrucciones $\rightarrow$ instruccionesAux
            \item instrucciones $\rightarrow \varepsilon$
            \item instruccionesAux $\rightarrow$ instruccionesAux ; instruccion 
            \item instruccionesAux $\rightarrow$ instruccion 
            \item instruccion  $\rightarrow$ @ expr
            \item instruccion  $\rightarrow$ if expr bloque
            \item instruccion  $\rightarrow$ if expr bloque else bloque
            \item instruccion  $\rightarrow$ while expr bloque
            \item instruccion  $\rightarrow$ read expr
            \item instruccion  $\rightarrow$ write expr
            \item instruccion  $\rightarrow$ nl
            \item instruccion  $\rightarrow$ new expr
            \item instruccion  $\rightarrow$ delete expr
            \item instruccion  $\rightarrow$ call identificador paramsReales
            \item paramsReales $\rightarrow$ ( paramsRealesAux )
            \item paramsRealesAux $\rightarrow$ paramsRealesLista
            \item paramsRealesAux $\rightarrow$ $\varepsilon$
            \item paramsRealesLista $\rightarrow$ paramsRealesLista , expr 
            \item paramsRealesLista $\rightarrow$ expr
        \end{itemize}
        Las siguientes expresiones se construyen de acuerdo a las siguientes prioridades (de menor a mayor prioridad):
        \begin{enumerate}
            \item Operador de asignación, es binario e infijo. 
            \item Operadores relacionales, son binarios e infijos.
            \item +, - (binario)
            \item and, or
            \item $^{\ast}$, /, \% son binarios e infijos.
            \item - (unario), not son binarios y prefijos.
            \item Operadores de indexación, de acceso a registro y de indirección. Unarios posfijos, asociativos.
        \end{enumerate}
        \begin{itemize}
            \item expr $\rightarrow$ e0
            \item e0 $\rightarrow$ e1 = e0
            \item e0 $\rightarrow$ e1
            \item e1 $\rightarrow$ e1 op1 e2
            \item e1 $\rightarrow$ e2
            \item e2 $\rightarrow$ e2 + e3
            \item e2 $\rightarrow$ e3 - e3
            \item e2 $\rightarrow$ e3
            \item e3 $\rightarrow$ e4 and e3
            \item e3 $\rightarrow$ e4 or e4
            \item e3 $\rightarrow$ e4
            \item e4 $\rightarrow$ e4 op4 e5
            \item e4 $\rightarrow$ e5
            \item e5 $\rightarrow$ op5 e5
            \item e5 $\rightarrow$ e6
            \item e6 $\rightarrow$ e6 op6
            \item e6 $\rightarrow$ ( e0 )
            \item e6 $\rightarrow$ literalEntero
            \item e6 $\rightarrow$ literalReal
            \item e6 $\rightarrow$ true
            \item e6 $\rightarrow$ false
            \item e6 $\rightarrow$ literalCadena
            \item e6 $\rightarrow$ identificador
            \item e6 $\rightarrow$ null
            \item op1 $\rightarrow$ $<$
            \item op1 $\rightarrow$ $>$
            \item op1 $\rightarrow$ $<=$
            \item op1 $\rightarrow$ $>=$
            \item op1 $\rightarrow$ ==
            \item op1 $\rightarrow$ !=
            \item op4 $\rightarrow$ *
            \item op4 $\rightarrow$ /
            \item op4 $\rightarrow$ \%
            \item op5 $\rightarrow$ -
            \item op5 $\rightarrow$ not
            \item op6 $\rightarrow$ [ expr ]
            \item op6 $\rightarrow$ . identificador
            \item op6 $\rightarrow$ \^{}
        \end{itemize}
        \subsubsection{Acondicionamiento de la gramática (descendente predictivo recursivo)}
        Por notación, la eliminación de recursión por izquierda será denotada comenzando el nombre por $rec$ y la factorización por $fac$.  
        \begin{itemize}
            \item programa $\rightarrow$ bloque
            \item bloque $\rightarrow$ \{ declaraciones instrucciones \}
        \end{itemize}
        \
        \begin{itemize}
            \item declaraciones $\rightarrow$ declaracionesAux \&\&
            \item declaraciones $\rightarrow \varepsilon$
            \item declaracionesAux $\rightarrow$ declaracion recDeclaracion
            \item recDeclaracion $\rightarrow$ ; declaracion recDeclaracion
            \item recDeclaracion $\rightarrow \varepsilon$
            \item declaracion $\rightarrow$ declaracionVar
            \item declaracion $\rightarrow$ declaracionTipo
            \item declaracion $\rightarrow$ declaracionProc
            \item declaracionVar $\rightarrow$ tipo0 identificador
            \item declaracionTipo $\rightarrow$ type tipo0 identificador
            \item declaracionProc $\rightarrow$ proc identificador paramsFormales bloque
            \item paramsFormales $\rightarrow$ ( paramsFormalesAux )
            \item paramsFormalesAux $\rightarrow$ paramsFormalesLista
            \item paramsFormalesAux $\rightarrow$ $\varepsilon$
            \item paramsFormalesLista $\rightarrow$ param recParamFormal
            \item recParamFormal $\rightarrow$ , param recParamFormal
            \item recParamFormal $\rightarrow \varepsilon$
            \item param $\rightarrow$ tipo0 referencia identificador
            \item referencia $\rightarrow$ \&
            \item referencia $\rightarrow \varepsilon$
        \end{itemize}
        \
        \begin{itemize}
            \item tipo0 $\rightarrow$ tipo1 recArray
            \item recArray $\rightarrow$ [ literalEntero ] recArray
            \item recArray $\rightarrow \varepsilon$
            \item tipo1 $\rightarrow$ \^{} tipo1
            \item tipo1 $\rightarrow$ tipo2
            \item tipo2 $\rightarrow$ struct \{ listaCampos \}
            \item listaCampos $\rightarrow$ campo recCampo
            \item recCampo $\rightarrow$ , campo recCampo
            \item recCampo $\rightarrow \varepsilon$
            \item campo $\rightarrow$ tipo0 identificador
            \item tipo2 $\rightarrow$ int
            \item tipo2 $\rightarrow$ real
            \item tipo2 $\rightarrow$ bool
            \item tipo2 $\rightarrow$ string
            \item tipo2 $\rightarrow$ identificador
        \end{itemize}
        \
        \begin{itemize}
            \item instrucciones $\rightarrow$ instruccionesAux
            \item instrucciones $\rightarrow \varepsilon$
            \item instruccionesAux $\rightarrow$ instruccion recInstruccion
            \item recInstruccion $\rightarrow$ ; instruccion recInstruccion
            \item recInstruccion $\rightarrow \varepsilon$
            \item instruccion $\rightarrow$ @ expr
            \item instruccion $\rightarrow$ if expr bloque facIf
            \item facIf $\rightarrow$ else bloque
            \item facIf $\rightarrow \varepsilon$
            \item instruccion $\rightarrow$ while expr bloque
            \item instruccion $\rightarrow$ read expr
            \item instruccion $\rightarrow$ write expr 
            \item instruccion $\rightarrow$ nl
            \item instruccion $\rightarrow$ new expr
            \item instruccion $\rightarrow$ delete expr
            \item instruccion $\rightarrow$ call identificador paramsReales
            \item instruccion $\rightarrow$ bloque
            \item paramsReales $\rightarrow$ ( paramsRealesAux )
            \item paramsRealesAux $\rightarrow$ paramsRealesLista
            \item paramsRealesAux $\rightarrow$ $\varepsilon$
            \item paramsRealesLista $\rightarrow$ expr recParamReal
            \item recParamReal $\rightarrow$ , expr recParamReal
            \item recParamReal $\rightarrow \varepsilon$
        \end{itemize}
        \
        \begin{itemize}
            \item expr $\rightarrow$ e0
            \item e0 $\rightarrow$ e1 facE1
            \item facE1 $\rightarrow$ = e0 facE1
            \item facE1 $\rightarrow \varepsilon$
            \item e1 $\rightarrow$ e2 recOp1
            \item recOp1 $\rightarrow$ op1 e2 recOp1
            \item recOp1 $\rightarrow \varepsilon$
            \item e2 $\rightarrow$ e3 facE3  recSuma 
            \item recSuma  $\rightarrow$ + e3 recSuma 
            \item recSuma  $\rightarrow \varepsilon$
            \item facE3  $\rightarrow$ - e3
            \item facE3  $\rightarrow \varepsilon$
            \item e3 $\rightarrow$ e4 facE4
            \item facE4 $\rightarrow$ and e3
            \item facE4 $\rightarrow$ or e4
            \item facE4 $\rightarrow \varepsilon$
            \item e4 $\rightarrow$ e5 recOp4
            \item recOp4 $\rightarrow$ op4 e5 recOp4
            \item recOp4 $\rightarrow \varepsilon$
            \item e5 $\rightarrow$ op5 e5
            \item e5 $\rightarrow$ e6
            \item e6 $\rightarrow$ e6Aux recOp6
            \item recOp6 $\rightarrow$ op6 recOp6
            \item recOp6 $\rightarrow \varepsilon$
            \item e6Aux  $\rightarrow$ ( e0 )
            \item e6Aux  $\rightarrow$ literalEntero
            \item e6Aux  $\rightarrow$ literalReal
            \item e6Aux  $\rightarrow$ true
            \item e6Aux  $\rightarrow$ false
            \item e6Aux  $\rightarrow$ literalCadena
            \item e6Aux  $\rightarrow$ identificador
            \item e6Aux  $\rightarrow$ null
            \item op1 $\rightarrow$ $<$
            \item op1 $\rightarrow$ $>$
            \item op1 $\rightarrow$ $<=$
            \item op1 $\rightarrow$ $>=$
            \item op1 $\rightarrow$ ==
            \item op1 $\rightarrow$ !=
            \item op4 $\rightarrow$ *
            \item op4 $\rightarrow$ /
            \item op4 $\rightarrow$ \%
            \item op5 $\rightarrow$ -
            \item op5 $\rightarrow$ not
            \item op6 $\rightarrow$ [ expr ]
            \item op6 $\rightarrow$ . identificador
            \item op6 $\rightarrow$ \^{}
        \end{itemize}
\end{document}